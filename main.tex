\documentclass[a4paper, 11pt]{article}

\usepackage[danish]{babel}
\usepackage[utf8]{inputenc}
%\usepackage{tgtermes}
%\usepackage{fouriernc}
\usepackage[T1]{fontenc}
\usepackage[margin=3cm]{geometry}

\usepackage{amssymb}
\usepackage{amsmath}
\usepackage{amsthm}
\usepackage{multicol}
\usepackage{xcolor}
\usepackage{wrapfig}

\usepackage{enumerate}
\usepackage[shortlabels]{enumitem}
\usepackage{verbatim}
\usepackage{hyperref}
\hypersetup{
    colorlinks=true,
    linkcolor=red,   
    urlcolor=red,
}
\newcommand{\N}{\mathbb{N}}
\newcommand{\Z}{\mathbb{Z}}
\newcommand{\Q}{\mathbb{Q}}
\newcommand{\R}{\mathbb{R}}
 
\title{{\large \textsc{Projekt Blodsukker}}}
\author{Cecilie Horshauge}
\date{\today}

\begin{document}
\maketitle
% Intro til opgaven
\noindent Hos raske mennesker reguleres blodsukkerkoncentrationen automatisk via leveren. 
Normalt ligger koncentrationen i intervallet 60-100 mg blodsukker pr. 100 ml blod. 
Hormonet insulin har den virkning, at det stimulerer cellernes optagelse af sukker fra blodet. 
Det følgende drejer sig om en forsøgsperson, der får en indsprøjtning af en bestemt dosis insulin.\\\\
Efter indsprøjtningen vil blodsukkerkoncentrationen ændre sig som funktion af den tid,
der er forløbet efter indsprøjtningen. Det antages i det følgende, at denne funktion med god tilnærmelse kan beskrives på følgende form: 
\[C(t)=83+73\cdot (a^t-b^t) \;\;\; \text{mg pr 100 mL}\]
hvor \(t\) er målt i timer og \(a=0.1\), \(b=0.5\)

\section*{Opgave A} 
Beskriv med ord hvorledes koncentrationen udvikler sig på kort, lidt længere og på helt langt sigt.

\subsection*{Løsning}

\section*{Opgave B} 
Hvad er den laveste koncentration, som forsøgspersonen oplever?
\subsection*{Løsning}

\section*{Opgave C} 
Hvornår falder koncentrationen hurtigst og hvornår stiger den hurtigst?
\subsection*{Løsning}

\section*{Opgave D} 
Hvornår vil funktionsværdien være inden for l\% fra sin asymptotiske værdi?
\subsection*{Løsning}

\section*{Opgave E} 
Man kan føle et stærkt ubehag, hvis blodsukkerkoncentrationen falder til under 60 mg pr. 100 ml. \\
I hvilket tidsrum efter indsprøjtningen må forsøgspersonen være forberedt på ubehag?
\subsection*{Løsning}

\end{document}